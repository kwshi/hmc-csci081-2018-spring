\documentclass{cs81-homework}

\title{Assignment 1}
\author{Kye W. Shi}
\date{2018 January 29 (Monday)}

\begin{document}

\begin{enumerate}
\item Prove constructively:
  \(A \land (B \lor C) \derives (A \land B) \lor (A \land C)\)

  \begin{solution}
  \end{solution}
  
\item Prove constructively: \(B \to A \derives (A \to C) \to (B \to C)\)

  \begin{solution}
  \end{solution}
  
\item Prove constructively: \(A \to \neg B \derives B \to \neg A\)
  
  \begin{solution}
  \end{solution}
  
\item Prove constructively: \(\neg \neg \neg A \derives \neg A\)
  
  \begin{solution}
  \end{solution}
  
\item Prove constructively: \(A \to \neg A \derives \neg A\)
  
  \begin{solution}
  \end{solution}
  
\item Prove classically: \(\neg A \to A \derives A\)
  
  \begin{solution}
  \end{solution}
  
\item Prove classically: \(\neg (\neg A \land \neg B) \derives A \lor B\)
  
  \begin{solution}
  \end{solution}
  
\item Prove constructively: \(A \lor B \derives \neg (\neg A \land \neg B)\)
  
  \begin{solution}
  \end{solution}
  
\item Note that the problems 7 and 8 are converses of each other.  How does this
  tell us that \(\lor\)Introduction cannot be used as the last step of the proof
  of \(\neg (\neg A \land \neg B) \derives A \lor B\)?
  
  \begin{solution}
  \end{solution}
  
\item Use propositional logic to prove this set-theoretic inclusion: If
  \(A \subseteq C\) and \(B \subseteq C\) then \(A \cup B \subseteq C\)?
  
  \begin{solution}
  \end{solution}
  
\item Use propositional logic to prove this set-theoretic identity:
  \[
    A \cap (B \cup C) = (A \cap B) \cup (A \cap C)
  \]
  
  \begin{solution}
  \end{solution}
  
\item State, in your own words, the difference in meanings of
  \(\Gamma \derives A\) vs. \(\Gamma \Derives A\), where \(\Gamma\) is a set of
  statements and \(A\) is a statement.
  
  \begin{solution}
  \end{solution}
  
\item Using a truth table, determine whether or not
  \[
    (A \to B) \to C \Derives (A \land B) \to C.
  \]
  
  \begin{solution}
  \end{solution}
  
\item Show that, in classical logic,
  \[
    \Gamma \derives A \text{ if, and only if, } \Gamma \cup \set{\neg A} \derives \bot
  \]
  
  \begin{solution}
  \end{solution}
  
\item Show that, in classical logic,
  \[
    \Gamma \Derives A \text{ if, and only if, } \Gamma \cup \set{\neg A} \Derives \bot
  \]
  
  \begin{solution}
  \end{solution}
  
\end{enumerate}

\end{document}