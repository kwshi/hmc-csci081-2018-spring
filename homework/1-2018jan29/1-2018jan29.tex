\documentclass{cs81-homework}

\title{Assignment 1}
\author{Kye W. Shi}
\date{2018 January 29 (Monday)}

\begin{document}

\begin{enumerate}
\item Prove constructively:
  \(A \and (B \or C) \derives (A \and B) \or (A \and C)\)

\item Prove constructively: \(B \to A \derives (A \to C) \to (B \to C)\)
\item Prove constructively: \(A \to \neg B \derives B \to \neg A\)



\item    Prove constructively: ¬¬¬A ⊢ ¬A
\item    Prove constructively: A→¬A ⊢ ¬A

\item    Prove classically: ¬A→A ⊢ A

\item    Prove classically: ¬(¬A ∧ ¬B) ⊢ A ∨ B
\item    Prove constructively: A∨B ⊢ ¬(¬A ∧ ¬B)

\item Note that the problems 7 and 8 are converses of each other.  How does this tell us that ∨ Introduction cannot be used as the last step of the proof of ¬(¬A ∧ ¬B) ⊢ A ∨ B?
\item  Use propositional logic to prove this set-theoretic inclusion: 
  If A ⊆ C and B ⊆ C then A⋃B ⊆ C
\item Use propositional logic to prove this set-theoretic identity: 

  A ⋂ (B ⋃ C) = (A ⋂ B) ⋃ (A ⋂ C)

\item State, in your own words, the difference in meanings of 𝚪 ⊢ A vs. 𝚪 ⊨A, where 𝚪 is a set of statements and A is a statement.

\item Using a truth table, determine whether or not 

  (A → B)→C ⊨ (A ∧ B)→C
\item Show that, in classical logic, 
  𝚪 ⊢ A if, and only if, 𝚪 ⋃ {¬A} ⊢ ⊥

\item Show that, in classical logic, 
  𝚪 ⊨ A if, and only if, 𝚪 ⋃ {¬A} ⊨ ⊥

\end{enumerate}
 


\end{document}