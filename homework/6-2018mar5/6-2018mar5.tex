\documentclass{cs81-homework}

\title{Assignment 6}
\author{}
\date{2018 March 5 (Monday)}

\begin{document}

\begin{introduction}
  This is the last assignment before the Midterm and Spring Break.

  \theintroduction

  Some problems ask to check your solutions using Prover9/Mace4. In these cases,
  paste in appropriate screen shots of the solution.  Download Prover9/Mace4
  here: \url{http://www.cs.unm.edu/~mccune/prover9/download/}.  I've created a
  quickstart guide to help you.  Also there are examples in the lecture slides.

  When using Mace4, you should use the Reformat menu to reformat to ``cooked''
  mode for ease in reading.
\end{introduction}

\begin{enumerate}

\item \points{10} Establish the satisfiability or unsatisfiability of the set of
  clauses below by the resolution method.  (Extend the table as needed.)  If a
  set is satisfiable, give an interpretation that satisfies it.
  
  \begin{solution}
    \begin{center}
      \begin{tabular}{|l||l|l|l|}
        \hline
        & Clause & Justification & Prover9 version \\
        \hline\hline
        1 & \(p \lor \lnot q \lor \lnot r\) & Given & \\ \hline
        2 & \(p \lor q \lor \lnot r \lor \lnot s\) & Given & \\ \hline
        3 & \(p \lor \lnot q \lor r \lor s\) & Given & \\ \hline
        4 & \(\lnot p \lor \lnot r \lor \lnot s\) & Given & \\ \hline
        5 & \(\lnot p \lor \lnot q \lor s\) & Given & \\ \hline
        6 & \(q \lor s\) & Given & \\ \hline
        7 & \( r \lor \lnot s\) & Given & \\
        \hline
      \end{tabular}
    \end{center}
  \end{solution}
  
\item[] \points{2/10} Once you have proved the statement by hand, check the result
  using Prover9/Mace4.  (Paste input and output here.)

  \begin{solution}
  \end{solution}
  
\item \points{10} Establish the satisfiability or unsatisfiability of the set of
  clauses below by the resolution method.  (Extend the tables as needed.)  If a
  set is satisfiable, give an interpretation that satisfies it.

  \begin{solution}
    \begin{center}
      \begin{tabular}{|l||l|l|l|}
        \hline
        & Clause & Justification & Prover9 version \\
        \hline\hline
        1 & \(\lnot p \lor q \lor r \lor s\) & Given & \\ \hline
        2 & \(\lnot p \lor \lnot q\) & Given & \\ \hline
        3 & \(q \lor \lnot s\) & Given & \\ \hline
        4 & \(\lnot q \lor r\) & Given & \\ \hline
        5 & \(\lnot r \lor s\) & Given & \\
        \hline
      \end{tabular}
    \end{center}
  \end{solution}
  
\item[] \points{2/10} Once you have proved the statement by hand, check the result
  using Prover9/Mace4.  (Paste input and output here.)

  \begin{solution}
  \end{solution}
  
\item \points{10} Determine whether the pair of atomic formulas below is
  unifiable by executing the unification algorithm on them.  Show the values of
  the set \(P\) and the unifier \(S\) as the algorithm iterates.  (Using a table
  as in the lecture slides is a good idea.)  If unifiable, give the most general
  unifier (MGU).  (In each case, the variables have already been renamed apart,
  so the renaming-apart step is not necessary.)  If not unifiable, state why
  unification fails.  In unifiable cases, check by direct substitution of the
  unifier into the formulae.
  \[
    \begin{array}{l@{}l@{}l@{}l@{}ll}
      P( & g(h(x)), & f(h(y)), & y, & x) & \text{vs.} \\
      P( & g(z), & f(z), & a, & b).
    \end{array}
  \]

  \begin{solution}
  \end{solution}
  
\item \points{10} Determine whether the pair of atomic formulae below is
  unifiable by executing the unification algorithm on them.  Show the values of
  the set \(P\) and the unifier \(S\) as the algorithm iterates.  (Using a table
  as in the lecture slides is a good idea.)  If unifiable, give the most general
  unifier (MGU).  (In each case, the variables have already been renamed apart,
  so the renaming-apart step is not necessary.)  If not unifiable, state why
  unification fails.  In unifiable cases, check by direct substitution of the
  unifier into the formulae.

  \[
    \begin{array}{l@{}l@{}l@{}l@{}ll}
      P( & g(h(x)), & f(h(y)), & y, & x) & \text{vs.} \\
      P( & g(z), & f(z), & h(a), & h(u)).
    \end{array}
  \]

  \begin{solution}
  \end{solution}
  
\item \points{5} Try to establish the satisfiability or unsatisfiability of the
  following set of clauses using resolution.  If a set is satisfiable, give an
  interpretation that satisfies it.  [Don't forget that variables do not carry
  across from one clause to another, but constants do.]  (Extend the table as
  necessary.)

  \begin{solution}
    \begin{center}
      \begin{tabular}{|l||l|l|}
        \hline
        & Clause & Justification \\
        \hline\hline
        1 & \(P(a)\) & Given \\ \hline
        2 & \(Q(y, x) \lor \lnot P(x)\) & Given \\ \hline
        3 & \(Q(b, x)\) & Given \\ \hline
        4 & \(\lnot Q(b, a)\) & Given \\
        \hline
      \end{tabular}
    \end{center}
  \end{solution}
  
\item[] \points{2/10} Once you have proved the statement by hand, check the result
  using Prover9/Mace4.  (Paste input and output here.)

  \begin{solution}
  \end{solution}
  
\item \points{10} Try to establish the satisfiability or unsatisfiability of the
  following set of clauses using resolution.  If a set is satisfiable, give an
  interpretation that satisfies it.  [Don't forget that variables do not carry
  across from one clause to another, but constants do.]  (Extend the table as
  necessary.)

  \begin{solution}
    \begin{center}
      \begin{tabular}{|l||l|l|}
        \hline
        & Clause & Justification \\
        \hline\hline
        1 & \(P(a)\) & Given \\
        2 & \(Q(y, x) \lor \lnot P(x)\) & Given \\
        3 & \(Q(b, x)\) & Given \\
        4 & \(\lnot Q(a, b)\) & Given \\
      \end{tabular}
    \end{center}
  \end{solution}
  
\item[] \points{2/10} Once you have proved the statement by hand, check the result
  using Prover9/Mace4.  (Paste input and output here.)

  \begin{solution}
  \end{solution}
  
\item \points{10} Try to establish the satisfiability or unsatisfiability of the
  following set of clauses using resolution.  If a set is satisfiable, give an
  interpretation that satisfies it.  [Don't forget that variables do not carry
  across from one clause to another, but constants do.]  (Extend the table as
  necessary.)

  \begin{solution}
    \begin{center}
      \begin{tabular}{|l||l|l|}
        \hline
        & Clause & Justification \\
        \hline\hline
        1 & \(Q(a)\) & Given \\
        2 & \(P(x) \lor \lnot Q(x)\) & Given \\
        3 & \(\lnot P(x) \lor Q(f(x))\) & Given \\
      \end{tabular}
    \end{center}
  \end{solution}
  
\item[] \points{2/10} Once you have proved the statement by hand, check the result
  using Prover9/Mace4.  (Paste input and output here.)

  \begin{solution}
  \end{solution}
  
\item \points{10} Prove the following sequent by the resolution method.  (Don't
  forget to negate the conclusion before converting to clausal form.)
  \[
    \forall x \: (P(x) \to Q(x)) \derives (\exists x \: P(x)) \to (\exists x \:
    Q(x)).
  \]

  \begin{solution}
  \end{solution}
  
\item[] \points{2/10} Once you have proved the statement by hand, check the result
  using Prover9/Mace4.  (Paste input and output here.)

  \begin{solution}
  \end{solution}
  
\item \points{10} Prove the following sequent by the resolution method.  (Don't
  forget to negate the conclusion before converting to clausal form.)
  \[
    (\exists x \: P(x)) \to (\forall x \: Q(x)) \derives (\forall x \: (P(x) \to
    Q(x))).
  \]

  \begin{solution}
  \end{solution}
  
\item[] \points{2/10} Once you have proved the statement by hand, check the result using
  Prover9/Mace4.  (Paste input and output here.)

  \begin{solution}
  \end{solution}
  
\item \points{10}
  Translate into predicate logic, then prove using resolution.
  
  Premises:
  \begin{enumerate}
  \item Every child loves candy.
  \item Anyone who loves candy is not a health nut.
  \item Anyone who drinks wheatgrass is a health nut.
  \item Dale drinks wheatgrass.
  \end{enumerate}
  
  Conclusion:
  \begin{enumerate}
  \item Dale is not a child.
  \end{enumerate}

  \begin{solution}
  \end{solution}
  
\item[] \points{2/10} Once you have proved the statement by hand, check the result
  using Prover9/Mace4.  (Paste input and output here.)  .

  \begin{solution}
  \end{solution}
  
\end{enumerate}

\end{document}